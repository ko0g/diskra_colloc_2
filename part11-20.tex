\section{Цепи и антицепи в частично упорядоченных множествах. Теорема Мирского. Теорема Дилуорса.}

$(P, \leq)$ -- ЧУМ

\subsection{Цепи и антицепи в частично упорядоченных множествах}

Цепь в $P$ -- $C \subseteq P: \forall x, y \in C: (x \leq y \lor y \leq x)$

Антицепь в $P$ -- $A \subseteq P: \forall x \neq y \in A: x, y$ -- несравнимы

$|A \cap C| \leq 1$ -- (если как минимум 2 элемента есть, то они не могут быть сравнимы и несравнимы одновременно)

\subsection{Теорема Мирского}
Наименьшее кол-во антицепей, покрывающих ЧУМ равно наибольшей длине цепи в этом ЧУМе

\subsection{Теорема Дилуорса}
Наименьшее кол-во цепей, покрывающих ЧУМ равно наибольшему размеру антицепи в этом ЧУМе

\section{LYM-неравенство, теорема Шпернера о размере максимальной антицепи в булевом кубе.}

\section{Граф сравнимости частично упорядоченного множества. Совершенные графы. Когда граф явля-
ется совершенным?}

\section{Вероятностное пространство, вероятностное распределение, примеры. Свойства вероятности. Пошаговое задание распределения, дерево событий}

Вероятностное пространство -- $(\underbrace{\Omega}_{\text{эл. исходы}}, \underbrace{P}_{\text{ф-я вероятности}})$

Функция вероятности/вероятностное распределение -- $P: \Omega \rightarrow [0, 1]$:
\[
\begin{cases}
    0 \leq P(\omega_i) \leq 1\\
    \sum_{i = 1}^{n} P(\omega_i) = 1
\end{cases}
\]
Событие $A$ -- подмн-во $A \subseteq \Omega$

Вероятность события $A$ -- $\sum P(\omega_i), \; \omega_i \in A$

Свойства вероятности:
\begin{enumerate}
    \item $P(\emptyset) = 0, P(\Omega) = 1$
    \item $P(A \sqcup B) = P(A) + P(B)$
    \item $P(\overline{A}) = 1 - P(A)$
    \item $A \subseteq B \Rightarrow P(A) \leq P(B)$
\end{enumerate}

Пошаговое задание распределения:

Если эксперимент состоит из нескольких шагов, распределение задаётся через дерево событий.

На каждом уровне дерева — возможные исходы текущего шага с условными вероятностями.

Вероятность пути в дереве -- произведение вероятностей вдоль рёбер пути.

Вероятность исхода -- вероятность соответствующего пути от корня до листа.

\section{Формула включений-исключений для вероятностей. Задача о беспорядках.}
\subsection{Формула включений-исключений для вероятностей}
$(\Omega, P): A_1, \hdots, A_n:$
\[
P(A_1 \cup \hdots \cup A_n) = \sum_{k = 1}^{n} (-1)^{k+1} \sum_{1 \leq i_1 < \hdots < i_k \leq n} P(A_{i_1} \cap \hdots \cap A_{i_k})
\]
\subsection{Задача о беспорядках}
Вероятностное пространство $\Omega$, $P = \frac{1}{n!}$ (все равновероятны)

Беспорядок $(p_1, \hdots, p_n)$ -- $\forall i: p_i \neq i$

Пусть $Y_i = \{p \in S_n \; | \; p_i = i\}$ -- $i$-ый элемент на своем месте

$P(\text{$p$ не беспорядок}) \Leftrightarrow \exists i: p \in Y_i \Leftrightarrow P(Y_1 \cup \hdots \cup Y_n)$

$P(Y_1 \cap \hdots \cap Y_k) = \frac{(n-k)!}{n!}$

$P(\text{$\geq 1$ неподвижной точки}) = \sum_{k = 1}^{n} (-1)^{k+1} \binom{n}{k} \cdot \frac{(n-k)!}{n!} = \sum_{k = 1}^{n} \frac{(-1)^{k+1}}{k!}$

\section{Условные вероятности, независимые события. Независимость событий в совокупности, отличие
от попарной независимости событий (приведите явный пример)}

Условная вероятность $P(A|B) = \frac{P(A \cap B)}{P(B)}$, $P(B) > 0$

Условная вероятность -- вероятность $A$ после $B$

События $A$ и $B$ независимые, если $P(A \cap B) = P(A) \cdot P(B)$

Независимость $A_1, \hdots, A_n$ в совокупности:
\[
\forall I \subseteq \{1, \hdots, n\}: P(\bigcap_{i \in I}A_i) = \prod_{i \in I} P(A_i)
\]
Попарная независимость $A_1, \hdots, A_n$:
\[
\forall i \neq j: P(A_i \cap A_j) = P(A_i) \cdot P(A_j)
\]
\\
Попарная независимость не означает независимость в совокупности

Пример:

\section{Формулы Байеса и полной вероятности. Пример с тестом на выявление болезни.}

\section{Случайная величина. Математическое ожидание случайной величины, линейность матожидания.
Дисперсия случайной величины.}

\section{Неравенства Маркова и Чебышёва.}
