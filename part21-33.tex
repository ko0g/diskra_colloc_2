\section{Независимые случайные величины. Математическое ожидание и дисперсия независимых случайных величин.}

Случайные величины $f, g$ независимые, если
\[
\forall x, y \in \mathbb{R}: P(f = x \land g = y) = P(f = x) \cdot P(g = y)
\]

Случ. величины $f_1, \hdots, f_n$ независимы $\Leftrightarrow$ независимы в совокупности

\subsection{Математическое ожидание и дисперсия независимых случайных величин.}

Пусть $f, g$ -- независимые случ. величины.

Мат. ожидание:
\[
\mathbb{E}(f \cdot g) = \mathbb{E}(f) \cdot \mathbb{E}(g)
\]
\vspace{-1cm}{
    \color{gray}
\subsection*{\textit{Доказательство:}}
\[
\E(f) = \sum_{x \in \mathbb{R}} x \cdot P(f = x), \E(g) = \sum_{x \in \mathbb{R}} x \cdot P(g = x), 
\]
\[
\E(f) \cdot \E(g) = \sum_{x \in \mathbb{R}} x \cdot P(f = x) \cdot \sum_{y \in \mathbb{R}} y \cdot P(g = y) = 
\]
\[
=\sum_{x, y \in \mathbb{R}} x \cdot y \cdot \overbrace{P(f = x) \cdot P(g = y)}^{P(f = x \cap g = y)} = \E(f \cdot g)
\]
}

Дисперсия:
\[
\mathbb{D}(f + g) = \mathbb{D}(f) + \mathbb{D}(g)
\]
\vspace{-1cm}{
    \color{gray}
\subsection*{\textit{Доказательство:}}
\[
\D(f+g) = \E((f+g)^2) - (\E(f+g))^2=
\]
\[
=\E(f^2 + 2fg + g^2) - (\E(f) + \E(g))^2 = 
\]
\[
=\E(f^2) - 2\E(f)\E(g) + \E(g^2) - (\E(f))^2 - 2\E(f)\E(g) - (\E(g))^2 = 
\]
\[
=\E(f^2)-(\E(f))^2 + \E(g^2) - (\E(g))^2 = \D(f) + \D(g)
\]
}

\section{Формула Стирлинга. Асимптотическое выражение для вероятности выпадения ровно половины
орлов при бросании честной монеты.}

\subsection{Формула Стирлинга}
\[
n! \sim \sqrt{2\pi n}\left(\frac{n}{e}\right)^n
\]
\subsection{Асимптотическое выражение для вероятности выпадения ровно половины
орлов при бросании честной монеты.}
Подбросим $n = 2k$ раз монету. Чему равна вероятность выпадения ровно $\frac{n}{2}$ орлов?
\[
P\left(\text{$=\frac{n}{2}$ орлов}\right) = \binom{n}{n/2} \cdot \left(\frac{1}{2}\right)^n
\]
Применим формулу Стирлинга:
\[
\frac{\binom{n}{n/2}}{2^n} = \frac{\frac{n!}{(n/2)!(n/2)!}}{2^n} \sim \frac{\sqrt{2 \pi n}\left(\frac{n}{e}\right)^n}{\pi n \left(\frac{n}{2e}\right)^n \cdot 2^n} = \frac{\sqrt{2 \pi n}}{\pi n} = \sqrt{\frac{2}{\pi n}}
\]

\section{Оценки для биномиальных коэффициентов.}
\[
\left(\frac{n}{k}\right)^k \leq \binom{n}{k} \leq \left(\frac{n \cdot e}{k}\right)^k
\]
\vspace{-1.5cm}{
    \color{gray}
\subsection*{\textit{Доказательство:}}
1) $\binom{n}{k} = \frac{n(n-1)\hdots(n-k+1)}{k(k-1)\hdots1} = \overbrace{\frac{n}{k} \cdot\frac{n-1}{k-1} \cdot \hdots \cdot \frac{n-k+1}{1}}^{k} \geq \left(\frac{n}{k}\right)^k$

При $a \geq b \geq 1: \frac{a}{b} \leq \frac{a-1}{b-1}$

2) $\forall x \in \mathbb{R}: e^x \geq 1 + x$

Покажем, что $\sum_{i = 0}^{k} \binom{n}{i} \leq \left(\frac{n \cdot e}{k}\right)^k$

Фикс. $t \in (0; 1)$:
\[
\sum_{i = 0}^{k} \binom{n}{i} \leq \sum_{i = 0}^{k} \binom{n}{i} \cdot t^{i-k} = \frac{1}{t^k} \sum_{i = 0}^{k} \binom{n}{i} \cdot t^i \leq \frac{1}{t^k} \sum_{i = 0}^{n} \binom{n}{i} \cdot t^i =
\]
\[
=\frac{1}{t^k}(1+t)^n
\]

При $t = \frac{k}{n}$: $\left(\frac{n}{k}\right)^k \cdot \underbrace{\left(1+\frac{k}{n}\right)^n}_{\rightarrow e} \Rightarrow \left(\frac{n}{k}\right)^k \cdot e^k = \left(\frac{n \cdot e}{k}\right)^k$
}

\section{Вероятностный метод: общая формулировка и оценка объединения. Нижняя оценка на диагональные числа Рамсея}

Пусть вер. пространство $(\Omega, P), f: \Omega \rightarrow \mathbb{R}$, $\mathbb{E}(f) = c$

Лемма:
\[
\begin{cases}
    \exists w_{min} \in \Omega: f(w_{min}) \leq c\\
    \exists w_{max} \in \Omega: f(w_{max}) \geq c\\
\end{cases}
\]
{
    \color{gray}
\subsection*{\textit{Доказательство:}}
\[
c = \E(f) = \sum_{i = 1}^{n} p_i \cdot f(w_i)
\]
От противного: $\forall w: f(w) > c$:
\[
\sum_{i = 1}^{n} p_i \cdot c < \sum_{i = 1}^{n} p_i \cdot f(w_i)
\]
\[
c < c
\]
Противоречие $\Rightarrow \exists w: f(w) \leq c$
}

\subsection{Пример применения вероятностного метода
для поиска разреза в графе величины более половины числа ребер}

Пусть $G = (V, E)$ -- граф. Тогда $\exists$ разрез $S \subseteq V$: $|S| \geq \frac{|E|}{2}$
{
    \color{gray}
\subsection*{\textit{Доказательство:}}
$\Omega$ -- все подмножества $n$ вершин ($|\Omega| = 2^n$). Все равновероятно

Пусть $f(S) = |E(S, V \setminus S)|$ -- случ величина

$\E(f) = \sum_{e \in E} \E(I_e)$:
\[
I_e = \begin{cases}
    1, \text{ если $e \in E(S, V \setminus S)$}\\
    0, \text{ иначе}
\end{cases}
\]
$
\E(I_e) = P(e \in E(S, V \setminus S)) = 2 \cdot \frac{1}{4} = \frac{1}{2}$ (все состояния вершин на концах ребра)
\[
\Downarrow
\]
\[
\E(f) = \frac{|E|}{2}
\]
\[
\Downarrow
\]
\begin{center}
    По лемме существует исход $S \in \Omega$ для которого $f(S) \geq \mathbb{E}(f)$
\end{center}
}

Оценка объединения:
\[
P\left(\bigcup_{i = 1}^{n} A_i\right) \leq \sum_{i = 1}^{n} P(A_i)
\]

\subsection{Нижняя оценка на диагональные числа Рамсея}
\[
R(k, k) \leq \binom{2k-1}{k-1} < 2^{2k-2}
\]
\[
\left[\frac{k \cdot 2^{k/2}}{2e}\right] < R(k, k), \quad \text{ при $k \geq 3$}
\]

\section{Производящие функции, определение. Их сложение и умножение на скаляр. Произведение производящих функций, основные свойства арифметических операций. Обратимые производящие
функции, примеры. Когда производящая функция обратима?}

$(a_0, a_1, a_2, \hdots), a_i \in \mathbb{C} \leftrightarrow A(x) = a_0 + a_1x + a_2x^2 + \hdots$

$A(x) = \sum_{n = 0}^{\infty} a_nx^n$ -- производящая функция (ПФ) $(a_0 ,a_1, \hdots)$

Переменная $x$ -- формальная (вместо нее может быть например \includegraphics[height=1.6em]{гусейн.jpg} )

Мн-во всех ПФ: $\mathbb{C}[[x]]$ (кольцо формальных степенных рядов)

Операции:
\begin{enumerate}
    \item $A(x) \pm B(x) = \sum_{n = 0}^{\infty} (a_n \pm b_n) x^n$
    \item $c \cdot A(x) = \sum_{n = 0}^{\infty} (c \cdot a_n)x^n$, $c$ -- const
    \item $A(x) \cdot B(x) = (a_0 + a_1x + \hdots)(b_0 + b_1x + \hdots) = \sum_{n = 0}^{\infty} c_nx^n$, где
    \[
    c_n = \sum_{k = 0}^{n} a_k \cdot b_{n-k}
    \]
\end{enumerate}

Константные ПФ: $c = (c, 0, 0, \hdots) \leftrightarrow C(x) = c$

Свойства операций:
\begin{enumerate}
    \item $(A(x) + B(x)) + C(x) = A(x) + (B(x) + C(x))$
    \item $A(x) + B(x) = B(x) + A(x)$
    \item $c \cdot A(x) = C(x) \cdot A(x)$, $c$ -- const
    \item $A(x) \cdot B(x) = B(x) \cdot A(x)$
    \item $A(x)(B(x) + C(x)) = A(x) B(x) + A(x) C(x)$
    \item $(A(x) \cdot B(x)) \cdot C(x) = A(x) \cdot (B(x) \cdot C(x))$
\end{enumerate}

ПФ $B(x)$ обратная к $A(x)$, если $A(x) \cdot B(x) = 1$, $\quad B(x) = A^{-1}(x)$

Пример: $A(x) = 1 + x + x^2 + \hdots$
\[
A(x) \cdot \overbrace{(1-x)}^{A^{-1}(x)} = 1 \Rightarrow \sum_{n = 0}^{\infty} x^n = \frac{1}{1-x}
\]

Свойства:
\begin{enumerate}
    \item $\frac{1}{A(x)} \cdot \frac{1}{B(x)} = \frac{1}{A(x) \cdot B(x)}$
    \item $\frac{A(x)}{B(x)} + \frac{C(x)}{D(x)} = \frac{A(x)D(x) + B(x)C(x)}{B(x)D(x)}$
\end{enumerate}
\text{}\\
ПФ $A(x)$ обратима $\Leftrightarrow a_0 \neq 0$

\section{Формальное дифференцирование производящих функций, свойства производной, правило Лейбница. Подстановка нуля в производящую функцию, вычисление $n$-го коффициента производящей
функции с использованием производных.}

\subsection{Формальное дифференцирование производящих функций, свойства производной, правило Лейбница}

Формальная производная ПФ $A(x) = \sum_{n = 0}^{\infty} a_nx^n$ -- ПФ $A'(x)$:
\[
A'(x) = \sum_{n = 1}^{\infty} n \cdot a_n x^{n-1}
\]
Свойства:
\begin{enumerate}
    \item $(A(x) \pm B(x))' = A'(x) \pm B'(x)$
    \item $(c \cdot A(x))' = c \cdot A'(x)$, $c$ -- const
    \item $(A(x) \cdot B(x))' = A'(x) \cdot B(x) + A(x) \cdot B'(x)$ -- правило Лейбница
\end{enumerate}
\subsection{Подстановка нуля в производящую функцию, вычисление $n$-го коффициента производящей
функции с использованием производных.}

Пусть $A(x) = \sum_{n = 0}^{\infty} a_nx^n$ -- ПФ:
\[
\begin{cases}
    A(0) = a_0\\
    A'(0) = a_1\\
    A''(0) = 2a_2
\end{cases}
\Rightarrow A^{(n)}(0) = n! \cdot a_n \Rightarrow a_n = \frac{A^{(n)}(0)}{n!} 
\]

Преф. суммы $A(x) = \sum_{n = 0}^{\infty} a_nx^n$:
\[
a_0 + (a_0 + a_1)x + (a_0 + a_1 + a_2)x^2 + \hdots = A(x) \cdot \frac{1}{1-x}
\]

\section{Связь произведения производящих функций с неупорядоченными выборками. Бином Ньютона,
обобщение на целые показатели.}

Пусть $S, T$ -- мн-ва, $S \cap T = \emptyset$

Пусть $A(x)$ -- ПФ неупорядоч. выборок из $S$, $B(x)$ -- ПФ неупорядоч. выборок из $T$. Тогда $A(x) \cdot B(x)$ -- ПФ неупорядоч. выборок из $S \cup T$

Примеры:
\begin{enumerate}
    \item Бином Ньютона:\\\\
    $\{a_1, \hdots, a_n\}: \binom{n}{k}$ -- способы выбрать $k$ элементов
    \[
    C(x) = \sum_{k = 0}^{\infty} \binom{n}{k} x^k = (1+x)^n \; \; | \; \; \{a_1, \hdots, a_n\} \leftrightarrow \overbrace{\underbrace{ \{a_1\} }_{1+x} \cup \hdots \cup \underbrace{ \{a_n\} }_{1+x}}^{(1+x)^n}
    \]
    \item Кол-во салатов:\\
    \begin{tabular}{@{}l|l@{}}
    \begin{minipage}[t]{0.4\linewidth}
    \begin{itemize}
        \item Перец -- 0 или 1
        \item Редиска -- $0,2,4,\hdots$
        \item Баклажан -- любое
        \item Помидор -- $\leq 3$
    \end{itemize}
    \end{minipage}
    &
    \begin{minipage}[t]{0.8\linewidth}
    Кол-во салатов из $n$ овощей = ?\\
    Перец: $1+ x$\\
    Редиска: $1+ x^2 + x^4 + \hdots = \frac{1}{1-x^2}$\\
    Баклажан: $1+ x + x^2 + \hdots = \frac{1}{1-x}$\\\\
    Помидор: $1+ x + x^2 + x^3$\\
    \end{minipage}
    \end{tabular}
    \[
    \Downarrow
    \]
    \begin{center}
        Ответ на задачу -- коэф. $[x^n]$:
        \[
        (1+ x)(1+ x^2 + x^4 + \hdots)(1+ x + x^2 + \hdots)(1+ x + x^2 + x^3)=
        \]
        \[
        =\frac{(1+x)(1+ x + x^2 + x^3)}{(1-x^2)(1-x)}
        \]
    \end{center}
\end{enumerate}

\subsection{Бином Ньютона,
обобщение на целые показатели.}
Пусть $\alpha \in \mathbb{C}$, $k \in \mathbb{N} \setminus \{0\}$:
\begin{itemize}
    \item $\binom{\alpha}{k} = \frac{\alpha(\alpha-1)\hdots(\alpha-k+1)}{k!}$, $\quad \binom{\alpha}{0} = 1$
    \item $\binom{-n}{k} = \frac{-n(-n-1)\hdots(-n-k+1)}{k!} = (-1)^k \binom{n+k-1}{k}$
\end{itemize}
\text{}
\\
Пусть $n \in \mathbb{N}$:
\[
(1+x)^{-n} = \sum_{k = 0}^{\infty} \binom{-n}{k} x^k
\]

\section{Линейные рекурентные соотношения с постоянными коэффициентами. Как устроена производящая функция последовательности, удовлетворяющей линейному рекурентному соотношению
с постоянными коэффициентами? Явная формула для общего члена такой последовательности,
метод ее вывода.}

$(a_0, a_1, \hdots)$ -- линейное рекурентное соотношение порядка $k$ с постоянными коэффициентами, если $\exists c_1, \hdots, \overbrace{c_k}^{\neq 0}: \forall n \geq 0:$
\[
a_{n+k} = c_1a_{n+k-1} + c_2a_{n+k-2} \hdots + c_ka_n
\]
Пример: $F_n = 1 \cdot F_{n-1} + 1 \cdot F_{n-2}$ -- Фибоначчи порядка 2

Через $c_1, \hdots, c_k$ и $a_0, \hdots, a_{k-1}$ однозначно определяется $a_n$

\subsection{Как устроена производящая функция последовательности, удовлетворяющей линейному рекурентному соотношению
с постоянными коэффициентами?}

Теорема: $A(x)$ -- ПФ линейной рекурентны порядка $k$. Тогда $\exists$ многочлены $P(x), Q(x) \in \mathbb{C}[x]: \deg P < k, \deg Q = k$:
\[
A(x) = \frac{P(x)}{Q(x)}
\]
\vspace{-1cm}{
\color{gray}
\subsection*{\textit{Доказательство:}}
Умножим $A(x)$ на $(c_1x + c_2x^2 + \hdots + c_kx^k)$:
\[
A(x) \cdot c_1x: a_0c_1x + a_1c_1x^2 + \hdots + a_{k-1}c_1x^k + \hdots
\]
\[
A(x) \cdot c_2x^2: a_0c_2x^2 + a_1c_2x^3 + \hdots + a_{k-2}c_2x^k + \hdots
\]
\[
\vdots
\]
\[
A(x) \cdot c_kx^k: a_0c_kx^k + \hdots
\]
Коэфф. при $[x^k]: c_1a_{k-1} + \hdots + c_ka_0$ -- рекурента
Слагаемые с $x^d, d < k$ -- $\overline{P}(x)$:
\[
A(x)(c_1x + c_2x^2 + \hdots + c_kx^k) = \overline{P}(x) + \sum_{t \geq k} b_tx^t=
\]
\[
=A(x) + \overbrace{\overline{P}(x) - \sum_{t < k} b_tx^t}^{P(x)}, \deg P < k
\]
\[
\Downarrow
\]
\[
A(x)\underbrace{(-1 + c_1x + \hdots + c_kx^k)}_{Q(x)} = P(x)
\]
}

\subsection{Явная формула для общего члена такой последовательности,
метод ее вывода.}

Пусть $a_1, \hdots, a_s$ -- различные корни со степенью вхождения $c_1, \hdots, c_s$. Разложим ПФ на сумму дробей:
\[
A(x) = \sum_{i = 1}^{s} \sum_{j = 1}^{c_i} \frac{B_{i, j}}{(1-a_ix)^{j}}
\]
Тогда по обобщенному биному Ньютона:
\[
[x^n]: \sum_{i = 1}^{s} \left(\sum_{j = 1}^{m_i} B_{i,j} \cdot \binom{-n}{j} \cdot (-a_i)^n\right)=
\]
\[
=\sum_{i = 1}^{s} \left(\sum_{j = 1}^{m_i} B_{i,j} \cdot \binom{n+j-1}{j-1} \cdot a_i^n\right)
\]
\section{Числа Фибоначчи: их производящая функция и явная формула (формула Бине).}

Пусть ПФ Фибоначии -- $F(x) = \sum_{n = 0}^{\infty} F_nx^n$:
\[
F_n = F_{n-1} + F_{n-2}, \quad F_0 = 0, F_1 = 1
\]
\[
\Downarrow
\]
\[
\sum_{n = 2}^{\infty} F_nx^n = \sum_{n = 2}^{\infty} F_{n-1}x^n + \sum_{n = 2}^{\infty} F_{n-2}x^n
\]
\[
F(x) - F_0 - F_1x = x(F(x) - F_0) + x^2 \cdot F(x)
\]
\[
F(x)(1 - x - x^2) = F_0 + F_1x - F_0x
\]
\[
F(x) = \frac{x}{1-x-x^2}
\]
Разложим на сумму дробей:
\[
F(x) = \frac{x}{1-x-x^2} = \frac{1}{\sqrt{5}} \left( \underbrace{\frac{1}{1 - \frac{1+\sqrt{5}}{2}x}}_{A} - \underbrace{\frac{1}{1 - \frac{1-\sqrt{5}}{2}x}}_{B} \right)
\]
\[
\begin{cases}
    A = \sum_{n = 0}^{\infty} \left(\frac{1 + \sqrt{5}}{2}\right)^n x^n\\
    B = \sum_{n = 0}^{\infty} \left(\frac{1 - \sqrt{5}}{2}\right)^n x^n
\end{cases}
\Rightarrow F_n = \frac{\left(\frac{1 + \sqrt{5}}{2}\right)^n - \left(\frac{1 - \sqrt{5}}{2}\right)^n}{\sqrt{5}}
\]
\section{Правильные скобочные последовательности. Критерий того, что скобочная последовательность
является правильной. Рекуррентная формула для числа правильных скобочных последовательностей.}
Правильные скобочные последовательности определяются следующим образом:
\begin{enumerate}
    \item $\emptyset$ -- ПСП
    \item П -- ПСП $\Rightarrow$ (П) -- ПСП
    \item П$_{1}$, П$_{2}$ -- ПСП $\Rightarrow$ П$_{1}$П$_{2}$ -- ПСП
\end{enumerate}

\subsection{Критерий того, что скобочная последовательность
является правильной.}

Последовательность из '(' и ')' -- ПСП, если и только если
\begin{enumerate}
    \item Кол-во '(' равно кол-ву ')'
    \item Разность кол-ва '(' и кол-ва ')' для любого префикса неотрицательна
\end{enumerate}

{
    \color{gray}
\subsection*{\textit{Доказательство:}}
$\Rightarrow$: (индукция по построению)

$\Leftarrow$: Полная индукция по длине слова: $\forall$ слово такого вида -- ПСП.

Переход ($\forall k < n \rightarrow n$):

Рассмотрим префикс min длины $k$, на котором кол-во '(' равно кол-ву ')':

\begin{center}
    (П$_1$)П$_2$ -- закрывающая скобка ')' на позиции $k$
\end{center}
К П$_2$ применяется предположение индукции . На позиции $k$ разность скобок равна 0 $\Rightarrow \forall i < k$ разность строго больше 0. Префикс П$_1$ -- префикс исходной минус '(' $\Rightarrow$ в П$_1$ для каждого префикса разность неотрицательна и в П$_1$ равно кол-во '(' и ')' $\Rightarrow$ П$_1$ -- ПСП
\[
\Downarrow
\]
\begin{center}
    (П$_1$)П$_2$ -- ПСП
\end{center}
}
Следствие: $\forall$ ПСП представляется в виде (П$_1$)П$_2$, причем ед. образом
\vspace{-1.5cm}{
    \color{gray}
\subsection*{\textit{Доказательство единственности:}}

}


\subsection{Рекуррентная формула для числа правильных скобочных последовательностей.}

Пусть $C_n$ -- кол-во ПСП с $n$ скобками '(':
\[
C_n = \sum_{k = 0}^{n - 1} C_k \cdot C_{n-k-1}, \quad C_0 = 1, C_1 = 1
\]
\vspace{-1.5cm}
{
    \color{gray}
\subsection*{\textit{Доказательство:}}
\begin{center}
    П = $\overbrace{(\underbrace{\text{П}_{1}}_{k})\underbrace{\text{П}_{2}}_{n - k - 1}}^{n}$
\end{center}

}

\section{Числа Каталана: их производящая функция и явная
формула}
Числа Каталана $C_n$:
\begin{itemize}
    \item Кол-во ПСП длины $2n$
    \item Кол-во способов соединить $2n$ точек на окружности непересек. хордами
\end{itemize}
\[
C_n = \sum_{k = 0}^{n - 1} C_k \cdot C_{n-k-1}, \quad C_0 = 1, C_1 = 1
\]

ПФ чисел Каталана:
\[
C(x) = \frac{1-\sqrt{1-4x}}{2x}
\]
{
    \color{gray}
\subsection*{\textit{Доказательство:}}
\[
C(x) = c_0 + c_1x + c_2x^2 + \hdots = \sum_{n = 0}^{\infty} c_nx^n
\]
\[
xC^2(x) = \sum_{n = 0}^{\infty}b_nx^{n+1}, \quad b_n = \sum_{k = 0}^{n} C_k C_{n-k} = C_{n+1}
\]
\[
\Downarrow
\]
\[
xC^2(x) = c_1x + c_2x^2 + \hdots = C(x) - 1
\]

\[
\Downarrow
\]
\[
xC^2(x) - C(x) + 1= 0
\]
Решим уравнение:
\[
xC^2(x) - C(x) + 1= x\left(C(x) - \frac{1}{2x}\right)^2 - \frac{1}{4x} + 1 = 
\]
\[
=x\left(C(x) - \frac{1}{2x}\right)^2 + \frac{4x-1}{4x} = 0
\]
\[
4x^2\left(C(x) - \frac{1}{2x}\right)^2 = 1 - 4x
\]
\[
\left(\underbrace{2xC(x) - 1}_{Y(x)}\right)^2 = 1 - 4x
\]
\[
Y^2(x) = 1-4x
\]
\[
Y(x) = \pm (2xC(x) - 1)
\]
Так как $C(0) = 1 \Rightarrow Y(x) = -(2xC(x)) - 1$:
\[
2xC(x) - 1 = -\sqrt{1-4x}
\]
\[
C(x) = \frac{1-\sqrt{1-4x}}{2x}
\]
}
\subsection{Явная формула}
\[
C_n = \frac{\binom{2n}{n}}{n+1}
\]
\vspace{-1.5cm}{
    \color{gray}
\subsection*{\textit{Доказательство:}}
\[
1-(1-4x)^{1/2} = 1 - \sum_{k = 0}^{\infty} \binom{1/2}{k} (-4x)^k=
\]
\[
-\sum_{k = 1}^{\infty} \frac{\frac{1}{2}(\frac{1}{2}-1)\hdots(\frac{1}{2}-k+1)}{k!} (-4x)^k =
\]
\[
=-\sum_{k = 1}^{\infty} \frac{1(-1)(-3)\hdots(-(2k-3))}{k! \cdot 2^k} (-4x)^k =
\]
\[
=\sum_{k = 1}^{\infty} \frac{1 \cdot 3 \cdot \hdots \cdot (2k-3)}{k! \cdot 2^k} \cdot 4^k \cdot x^k=
\]
\[
=\sum_{k = 1}^{\infty} \frac{(2k-2)! \cdot 4^k}{k! \cdot 2^k \cdot 2 \cdot 4 \cdot \hdots \cdot (2k-2)}x^k = 
\]
\[
=\sum_{k = 1}^{\infty} \frac{2}{k} \cdot \frac{(2k-2)!}{(k-1)!(k-1!)} x^k = 
\]
\[
=\sum_{k = 1}^{\infty} \frac{2}{k} \cdot \binom{2k-2}{k-1} x^k
\]
\[
\Downarrow
\]
\[
C(x) = \sum_{k = 1}^{\infty} \frac{1}{k} \binom{2k-2}{k-1} x^{k-1} = 
\]
\[
=\sum_{n = 0}^{\infty} \frac{1}{n+1} \binom{2n}{n}x^n
\]
\[
\Downarrow
\]
\[
C_n = \frac{1}{n+1} \binom{2n}{n}
\]
}