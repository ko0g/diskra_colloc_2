\section{Независимые случайные величины. Математическое ожидание и дисперсия независимых случайных величин.}

Случайные величины $f, g$ независимые, если
\[
\forall x, y \in \mathbb{R}: P(f = x \land g = y) = P(f = x) \cdot P(g = y)
\]

Случ. величины $f_1, \hdots, f_n$ независимы $\Leftrightarrow$ независимы в совокупности

\subsection{Математическое ожидание и дисперсия независимых случайных величин.}

Пусть $f, g$ -- независимые случ. величины.

Мат. ожидание:
\[
\mathbb{E}(f \cdot g) = \mathbb{E}(f) \cdot \mathbb{E}(g)
\]

Дисперсия:
\[
\mathbb{D}(f + g) = \mathbb{D}(f) + \mathbb{D}(g)
\]

\section{Формула Стирлинга. Асимптотическое выражение для вероятности выпадения ровно половины
орлов при бросании честной монеты.}

\subsection{Формула Стирлинга}
\[
n! \sim \sqrt{2\pi n}\left(\frac{n}{e}\right)^n
\]
\subsection{Асимптотическое выражение для вероятности выпадения ровно половины
орлов при бросании честной монеты.}
Подбросим $n = 2k$ раз монету. Чему равна вероятность выпадения ровно $\frac{n}{2}$ орлов?
\[
P\left(\text{$=\frac{n}{2}$ орлов}\right) = \binom{n}{n/2} \cdot \left(\frac{1}{2}\right)^n
\]
Применим формулу Стирлинга:
\[
\frac{\binom{n}{n/2}}{2^n} = \frac{\frac{n!}{(n/2)!(n/2)!}}{2^n} \sim \frac{\sqrt{2 \pi n}\left(\frac{n}{e}\right)^n}{\pi n \left(\frac{n}{2e}\right)^n \cdot 2^n} = \frac{\sqrt{2 \pi n}}{\pi n} = \sqrt{\frac{2}{\pi n}}
\]

\section{Оценки для биномиальных коэффициентов.}
\[
\left(\frac{n}{k}\right)^k \leq \binom{n}{k} \leq \left(\frac{n \cdot e}{k}\right)^k
\]

\section{Вероятностный метод: общая формулировка и оценка объединения. Нижняя оценка на диагональные числа Рамсея}

Пусть вер. пространство $(\Omega, P), f: \Omega \rightarrow \mathbb{R}$, $\mathbb{E}(f) = c$

Лемма:
\[
\begin{cases}
    \exists w_{min} \in \Omega: f(w_{min}) \leq c\\
    \exists w_{max} \in \Omega: f(w_{max}) \geq c\\
\end{cases}
\]

Оценка объединения:
\[
P\left(\bigcup_{i = 1}^{n} A_i\right) \leq \sum_{i = 1}^{n} P(A_i)
\]

\subsection{Нижняя оценка на диагональные числа Рамсея}
\[
R(k, k) \leq \binom{2k-1}{k-1} < 2^{2k-2}
\]
\[
\left[\frac{k \cdot 2^{k/2}}{2e}\right] < R(k, k), \quad \text{ при $k \geq 3$}
\]

\section{Производящие функции, определение. Их сложение и умножение на скаляр. Произведение производящих функций, основные свойства арифметических операций. Обратимые производящие
функции, примеры. Когда производящая функция обратима?}

$(a_0, a_1, a_2, \hdots), a_i \in \mathbb{C} \leftrightarrow A(x) = a_0 + a_1x + a_2x^2 + \hdots$

$A(x) = \sum_{n = 0}^{\infty} a_nx^n$ -- производящая функция (ПФ) $(a_0 ,a_1, \hdots)$

Переменная $x$ -- формальная (вместо нее может быть например \includegraphics[height=1.6em]{гусейн.jpg} )

Мн-во всех ПФ: $\mathbb{C}[[x]]$ (кольцо формальных степенных рядов)

Операции:
\begin{enumerate}
    \item $A(x) \pm B(x) = \sum_{n = 0}^{\infty} (a_n \pm b_n) x^n$
    \item $c \cdot A(x) = \sum_{n = 0}^{\infty} (c \cdot a_n)x^n$, $c$ -- const
    \item $A(x) \cdot B(x) = (a_0 + a_1x + \hdots)(b_0 + b_1x + \hdots) = \sum_{n = 0}^{\infty} c_nx^n$, где
    \[
    c_n = \sum_{k = 0}^{n} a_k \cdot b_{n-k}
    \]
\end{enumerate}

Константные ПФ: $c = (c, 0, 0, \hdots) \leftrightarrow C(x) = c$

Свойства операций:
\begin{enumerate}
    \item $(A(x) + B(x)) + C(x) = A(x) + (B(x) + C(x))$
    \item $A(x) + B(x) = B(x) + A(x)$
    \item $c \cdot A(x) = C(x) \cdot A(x)$, $c$ -- const
    \item $A(x) \cdot B(x) = B(x) \cdot A(x)$
    \item $A(x)(B(x) + C(x)) = A(x) B(x) + A(x) C(x)$
    \item $(A(x) \cdot B(x)) \cdot C(x) = A(x) \cdot (B(x) \cdot C(x))$
\end{enumerate}

ПФ $B(x)$ обратная к $A(x)$, если $A(x) \cdot B(x) = 1$, $\quad B(x) = A^{-1}(x)$

Пример: $A(x) = 1 + x + x^2 + \hdots$
\[
A(x) \cdot \overbrace{(1-x)}^{A^{-1}(x)} = 1 \Rightarrow \sum_{n = 0}^{\infty} x^n = \frac{1}{1-x}
\]

Свойства:
\begin{enumerate}
    \item $\frac{1}{A(x)} \cdot \frac{1}{B(x)} = \frac{1}{A(x) \cdot B(x)}$
    \item $\frac{A(x)}{B(x)} + \frac{C(x)}{D(x)} = \frac{A(x)D(x) + B(x)C(x)}{B(x)D(x)}$
\end{enumerate}
\text{}\\
ПФ $A(x)$ обратима $\Leftrightarrow a_0 \neq 0$

\section{Формальное дифференцирование производящих функций, свойства производной, правило Лейбница. Подстановка нуля в производящую функцию, вычисление $n$-го коффициента производящей
функции с использованием производных.}

\subsection{Формальное дифференцирование производящих функций, свойства производной, правило Лейбница}

Формальная производная ПФ $A(x) = \sum_{n = 0}^{\infty} a_nx^n$ -- ПФ $A'(x)$:
\[
A'(x) = \sum_{n = 1}^{\infty} n \cdot a_n x^{n-1}
\]
Свойства:
\begin{enumerate}
    \item $(A(x) \pm B(x))' = A'(x) \pm B'(x)$
    \item $(c \cdot A(x))' = c \cdot A'(x)$, $c$ -- const
    \item $(A(x) \cdot B(x))' = A'(x) \cdot B(x) + A(x) \cdot B'(x)$ -- правило Лейбница
\end{enumerate}
\subsection{Подстановка нуля в производящую функцию, вычисление $n$-го коффициента производящей
функции с использованием производных.}

Пусть $A(x) = \sum_{n = 0}^{\infty} a_nx^n$ -- ПФ:
\[
\begin{cases}
    A(0) = a_0\\
    A'(0) = a_1\\
    A''(0) = 2a_2
\end{cases}
\Rightarrow A^{(n)}(0) = n! \cdot a_n \Rightarrow a_n = \frac{A^{(n)}(0)}{n!} 
\]

Преф. суммы $A(x) = \sum_{n = 0}^{\infty} a_nx^n$:
\[
a_0 + (a_0 + a_1)x + (a_0 + a_1 + a_2)x^2 + \hdots = A(x) \cdot \frac{1}{1-x}
\]

\section{Связь произведения производящих функций с неупорядоченными выборками. Бином Ньютона,
обобщение на целые показатели.}

Пусть $S, T$ -- мн-ва, $S \cap T = \emptyset$

Пусть $A(x)$ -- ПФ неупорядоч. выборок из $S$, $B(x)$ -- ПФ неупорядоч. выборок из $T$. Тогда $A(x) \cdot B(x)$ -- ПФ неупорядоч. выборок из $S \cup T$

Примеры:
\begin{enumerate}
    \item Бином Ньютона:\\\\
    $\{a_1, \hdots, a_n\}: \binom{n}{k}$ -- способы выбрать $k$ элементов
    \[
    C(x) = \sum_{k = 0}^{\infty} \binom{n}{k} x^k = (1+x)^n \; \; | \; \; \{a_1, \hdots, a_n\} \leftrightarrow \overbrace{\underbrace{ \{a_1\} }_{1+x} \cup \hdots \cup \underbrace{ \{a_n\} }_{1+x}}^{(1+x)^n}
    \]
    \item Кол-во салатов:\\
    \begin{tabular}{@{}l|l@{}}
    \begin{minipage}[t]{0.4\linewidth}
    \begin{itemize}
        \item Перец -- 0 или 1
        \item Редиска -- $0,2,4,\hdots$
        \item Баклажан -- любое
        \item Помидор -- $\leq 3$
    \end{itemize}
    \end{minipage}
    &
    \begin{minipage}[t]{0.8\linewidth}
    Кол-во салатов из $n$ овощей = ?\\
    Перец: $1+ x$\\
    Редиска: $1+ x^2 + x^4 + \hdots = \frac{1}{1-x^2}$\\
    Баклажан: $1+ x + x^2 + \hdots = \frac{1}{1-x}$\\\\
    Помидор: $1+ x + x^2 + x^3$\\
    \end{minipage}
    \end{tabular}
    \[
    \Downarrow
    \]
    \begin{center}
        Ответ на задачу -- коэф. $[x^n]$:
        \[
        (1+ x)(1+ x^2 + x^4 + \hdots)(1+ x + x^2 + \hdots)(1+ x + x^2 + x^3)=
        \]
        \[
        =\frac{(1+x)(1+ x + x^2 + x^3)}{(1-x^2)(1-x)}
        \]
    \end{center}
\end{enumerate}

\section{Линейные рекурентные соотношения с постоянными коэффициентами. Как устроена производящая функция последовательности, удовлетворяющей линейному рекурентному соотношению
с постоянными коэффициентами? Явная формула для общего члена такой последовательности,
метод ее вывода.}

$(a_0, a_1, \hdots)$ -- линейное рекурентное соотношение порядка $k$ с постоянными коэффициентами, если $\exists c_1, \hdots, \overbrace{c_k}^{\neq 0}: \forall n \geq 0:$
\[
a_{n+k} = c_1a_{n+k-1} + c_2a_{n+k-2} \hdots + c_ka_n
\]
Пример: $F_n = 1 \cdot F_{n-1} + 1 \cdot F_{n-2}$ -- Фибоначчи порядка 2

Через $c_1, \hdots, c_k$ и $a_0, \hdots, a_{k-1}$ однозначно определяется $a_n$

\subsection{Как устроена производящая функция последовательности, удовлетворяющей линейному рекурентному соотношению
с постоянными коэффициентами?}

Теорема: $A(x)$ -- ПФ линейной рекурентны порядка $k$. Тогда $\exists$ многочлены $P(x), Q(x) \in \mathbb{C}[x]: \deg P < k, \deg Q = k$:
\[
A(x) = \frac{P(x)}{Q(x)}
\]