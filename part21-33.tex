\section{Независимые случайные величины. Математическое ожидание и дисперсия независимых случайных величин.}

Случайные величины $f, g$ независимые, если
\[
\forall x, y \in \mathbb{R}: P(f = x \land g = y) = P(f = x) \cdot P(g = y)
\]

Случ. величины $f_1, \hdots, f_n$ независимы $\Leftrightarrow$ независимы в совокупности

\subsection{Математическое ожидание и дисперсия независимых случайных величин.}

Пусть $f, g$ -- независимые случ. величины.

Мат. ожидание:
\[
\mathbb{E}(f \cdot g) = \mathbb{E}(f) \cdot \mathbb{E}(g)
\]

Дисперсия:
\[
\mathbb{D}(f + g) = \mathbb{D}(f) + \mathbb{D}(g)
\]

\section{Формула Стирлинга. Асимптотическое выражение для вероятности выпадения ровно половины
орлов при бросании честной монеты.}

\subsection{Формула Стирлинга}
\[
n! \sim \sqrt{2\pi n}\left(\frac{n}{e}\right)^n
\]
\subsection{Асимптотическое выражение для вероятности выпадения ровно половины
орлов при бросании честной монеты.}
Подбросим $n = 2k$ раз монету. Чему равна вероятность выпадения ровно $\frac{n}{2}$ орлов?
\[
P\left(\text{$=\frac{n}{2}$ орлов}\right) = \binom{n}{n/2} \cdot \left(\frac{1}{2}\right)^n
\]
Применим формулу Стирлинга:
\[
\frac{\binom{n}{n/2}}{2^n} = \frac{\frac{n!}{(n/2)!(n/2)!}}{2^n} \sim \frac{\sqrt{2 \pi n}\left(\frac{n}{e}\right)^n}{\pi n \left(\frac{n}{2e}\right)^n \cdot 2^n} = \frac{\sqrt{2 \pi n}}{\pi n} = \sqrt{\frac{2}{\pi n}}
\]

\section{Оценки для биномиальных коэффициентов.}
\[
\left(\frac{n}{k}\right)^k \leq \binom{n}{k} \leq \left(\frac{n \cdot e}{k}\right)^k
\]